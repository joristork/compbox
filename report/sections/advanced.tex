In instruction parsing all instructions are parsed to find out to which
register they write and which registers they need. With this information,
dataflow analysis and liveness analysis can be done. Noteworthy is that both
analysis implementations take the double precision floating point 
operations into account. This means, that two consecutive registers are used when 
appropriate.
\subsubsection{Dataflow analysis}
With a control flow graph (CFG) object we can create a dataflow object. When a 
dataflow object is created, first the gen set of each block is determined and 
set in the basicblock. After this, the reach of each block has to be calculated
(which other blocks can be reached using the directed edges of the graph). With 
this information, we can generate the kill set for each block.\\
\\
Now that we have the gen- and kill-set of each block, we can calculate the
in and out sets using the reaching definitions algorithm.\\
\\
An overview of the sets and the number of iterations needed for the reaching 
definitions algorithm can be printed by calling a print function on the dataflow
object.
\subsubsection{Liveness analysis}
To optimise our assembly further, it is important to know which variables are
alive at which point. That is: if a instruction writes to a register, the
register is alive between the write and the last read (without writing in 
between). Wikipedia: "a variable is live if it holds a value that may be needed 
in the future". \\
First, we start by determining the $live_{out}$ set of each block. These
are the registers that are needed somewhere in the code after the block. It 
is therefore crucial to examine all edges. \\
The $live_{in} $ sets are all variables that are needed in the block or further
along in the graph, but are not set before they are needed in the block.
For each instruction in the block we determine which registers are needed. This 
is done backwards. We can search in the block which instruction writes the last
to these registers. This instruction is saved in a set called $kill$. The 
analysis takes into account that some jump functions have a elaborate set of
registers they need. This is the case for the \texttt{jal}, \texttt{jal} and 
\texttt{j \$31} instructions. These instructions are function calls.
In the case of a \texttt{j \$31} instruction,\\
\$fp, \$sp, \$16-\$23, \$f20, \$f22, \$f24, \$f26, \$f28, \$f30, \$2, \$3, \$f0, \$f1, \$f2, \$f3 \\are alive. \\
In the case of a \texttt{jal} or a \texttt{jalr} instruction,\\
\$4-\$7 en \$f12-\$f15 \\
are alive and \\
\$31 \\
is set.

\subsubsection{Liveness optimalisation}
A small optimalisation is done based on our liveness analysis. We examine each
instruction that writes for liveness of the written register. If the register 
is never used (in the graph), it can be removed.
