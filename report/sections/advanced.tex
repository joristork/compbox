In instruction parsing all instructions are parsed to find out to which
register they write and which registers they need. With this information,
dataflow analysis and liveness analysis can be done. Noteworthy is that both
analysis implementations take the double precision floating point 
operations into account. This means, that two consecutive registers are used when 
appropriate.
\subsubsection{Dataflow analysis}
With a control flow graph (CFG) object we can create a dataflow object. When a 
dataflow object is created, first the gen set of each block is determined and 
set in the basicblock. After this, the reach of each block has to be calculated
(which other blocks can be reached using the directed edges of the graph). With 
this information, we can generate the kill set for each block.

Now that we have the gen- and kill-set of each block, we can calculate the
in and out sets using the reaching definitions algorithm.

A overview of the sets and the number of iterations needed for the reaching 
definitions algorithm can be printed by calling a print function on the dataflow
object.
\subsubsection{Liveness analysis}
For all blocks the $\text{LIVE}_\text{in}$ and $\text{LIVE}_\text{out}$ sets are determined.
The $\text{LIVE}_\text{in}$ set is determined by:
\[
\text{LIVE}_\text{in}[block] = \bigcup_{i \in \text{block}}(used\_regs[i]) - (registers\_set\_in\_block\_before[i]).
\]
The $\text{LIVE}_\text{out}$ set by:
\begin{align*}
\text{LIVE}_\text{out}[\text{block}]		&= \bigcup_{p \in succ[\text{block}]} \text{LIVE}_\text{in}[p],\\
\text{LIVE}_\text{out}[\text{finalblock}]	&= \emptyset.
\end{align*}
If a \texttt{jal} or \texttt{jalr} instruction is encountered, it is not clear 
anymore which registers are used in the function the jump points to. We will 
have to assume the worst and set all registers minus the registers that
are set before the jump as $\text{LIVE}_\text{in}$ set for that block.
