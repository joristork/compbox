\lstset{ %
language=Python,                % the language of the code
}

For instruction parsing we use Python Lex-Yacc (PLY), an implementation of Lex and Yacc for Python. The Lex part of PLY splits the source in \emph{tokens}, e.g. a \verb!COMMENT!, or a comma. These tokens are passed to Yacc, which finds hierarchical structure in them. When Yacc recognizes a certain pattern, a corresponding action is taken.

\subsection{Lex}

Our lexer splits the source in six tokens: \verb!COMMENT! for comments, \verb!RAW! for compiler directives (lines starting with a dot), \verb!HEX! for hexadecimal values, \verb!INT! for integer values (unsigned), \verb!ID! for label names and instruction names, and \verb!REGISTER! for register identifiers (\verb!$1!, \verb!$2!, etc.)
The following regular expressions specify the tokens.
\begin{quote}
\begin{verbatim}
COMMENT  : #.*
RAW      : \..+
HEX      : 0x[0-9a-f]+
INT      : [0-9]+
ID       : ($L){0,1}[_a-zA-Z0-9][_a-zA-Z0-9\.]*
REGISTER : $[a-z0-9]+
\end{verbatim}
\end{quote}
In addition, we consider the characters `\verb!,!', `\verb!(!', `\verb!)!', `\verb!:!', `\verb!.!', `\verb!+!', and `\verb!-!' as individual tokens, called \emph{literals}.

\subsection{Yacc}
We specified the structure of the assembly language according to the following grammar rules.

\begin{grammar}
<expr> := <raw>
\alt <label>
\alt <comment>
\alt <instr>
\alt <instr> <comment>

<raw> := \verb!RAW!

<label> := \verb!ID! `:'

<comment> := \verb!COMMENT!

<instr> := \verb!ID!
\alt \verb!ID! <arg>
\alt \verb!ID! <arg> `,' <arg>
\alt \verb!ID! <arg> `,' <arg> `,' <arg>

<arg> := <int>
\alt <hex>
\alt <register>
\alt \verb!ID!
\alt \verb!ID! `+' <arg>
\alt \verb!ID! `-' <arg>
\alt <int> `(' <register> `)'
\alt \verb!ID!  `(' <register> `)'

<int> := \verb!INT!
\alt `-' \verb!INT!

<hex> := \verb!HEX!

<register> := \verb!REGISTER!
\end{grammar}

Each expression is stored in a Python object of type \verb!Raw!, \verb!Label!, \verb!Comment!, \verb!Instr! or \verb!Register!.

\subsection{Registers}
To perform dataflow analysis and liveness analysis, we need to parse each 
instruction to determine which registers are set and which registers are used. 
All instructions have a different format, so they need to be parsed seperatly. 
This process is done in the \code{parse\_instr} module. When we run \code{parse\_instr.parse},
it parses each Instr object from the \code{ir} module and sets the register that the
instructoin writes to, the registers that it needs, the offset for writing to 
memory, and immidiate values. An example:
\begin{lstlisting}
if ins.instr == 'add':  
    if len(ins.args) == 3:
        self.gen = [ins.args[0]]
        if self.is_reg(ins.args[2]):
            self.need = [ins.args[1], ins.args[2]]
        else: 
            self.need = [ins.args[1]]
            self.ival = ins.args[2]
    else:
        raise Exception("Invalid number of args for ins: ", ins.instr)   
\end{lstlisting}
