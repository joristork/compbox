\lstset{ %
language=[mips]Assembler,         % the language of the code
}

The flat module contains five different optimalisation functions that remove
redundant code on a global basis. In this subsection we will elaborate on those 
functions. These functions perform better when comments are removed so there is
a function that removes all comments from the code. We decided to not lose 
information in the assembly code.
\subsubsection{Jump - code - Label}
When there is a jump, every code that is between the jump and the label is 
never reached. Thus: the instruction between a jump and a label can be removed.
This function probably will not speed up the code, but can reduce
lines in the assembly code.
Example:
\begin{lstlisting}
    addu    $2,$3,4
    lw      $3,60($fp)
    j       $L4
    mtc1    $2,$f0
    div.d   $f0,$f0,$f2	
$L6:
    jal     random 
\end{lstlisting}
Becomes:
\begin{lstlisting}
    addu    $2,$3,4
    lw      $3,60($fp)
    j       $L4
$L6:
    jal	    random 
\end{lstlisting}
\subsubsection{Label - Label}
When two labels placed directly after eachother, one of the two is redundant.
The second label is removed and each jump in the code that jumps to the second 
label is changed to jump to the first label. This function can reduce lines and
basicblocks for later on in optimalisation. In combination with the jump 
optimalisation "useless branches" this can potentially speed up the program.
Example:
\begin{lstlisting}
    addu    $2,$3,4
    lw      $3,60($fp)
    bne     $2,$0,$L6
    mtc1    $2,$f0
    div.d   $f0,$f0,$f2	
$L5:
$L6:
    jal	    random 
\end{lstlisting}
Becomes:
\begin{lstlisting}
    addu    $2,$3,4
    lw      $3,60($fp)
    bne     $2,$0,$L5
    mtc1    $2,$f0
    div.d   $f0,$f0,$f2	
$L5:
    jal     random 
\end{lstlisting}
\subsubsection{Label - Jump}
If a jump comes directly after a label, both are redundant. We can remove both
lines and adjust all control instructions that point to the label so that they
point to the label the jump was pointing to. If this situation occurs, two
lines of assembly can be removed and the program is optimised by removing one 
jump.
Example:
\begin{lstlisting}
    addu    $2,$3,4
    bne     $2,$0,$L5
    mtc1    $2,$f0
$L5:
    j       $L8
$L6:
    div.d   $f0,$f0,$f2	
\end{lstlisting}
Becomes:
\begin{lstlisting}
    addu    $2,$3,4
    bne     $2,$0,$L8
    mtc1    $2,$f0
$L6:
    div.d   $f0,$f0,$f2	
\end{lstlisting}
\subsubsection{Useless branch/jump}
If a control function jumps to a label that comes directly after the jump/branch
, the instruction can be removed. This results in a smaller assembly file and
a faster program.
Example:
\begin{lstlisting}
    addu    $2,$3,4
    bne     $2,$0,$L6
$L6:
    div.d   $f0,$f0,$f2	
\end{lstlisting}
Becomes:
\begin{lstlisting}
    addu    $2,$3,4
$L6:
    div.d   $f0,$f0,$f2	
\end{lstlisting}
\subsubsection{Branch - Jump - Label}
This function removes a jump when we find a branch - jump - label
combination where the branch points to the label. We can optimise this
by negating the brench, making it point to the jump adres and removing the
label. This results in a smaller assembly file and
a faster program.
Example:
\begin{lstlisting}
    slt     $2,$2,$3
    bne     $2,$0,$L6
    j       $L4
$L6:
    jal     random
\end{lstlisting}
Becomes:
\begin{lstlisting}
    slt     $2,$2,$3
    beq     $2,$0,$L4
$L6:
    jal     random
\end{lstlisting}
